\chapter{Analysis}

\section{Introduction}

\subsection{Client Identification} 
My client is my ICT teacher at Long Road
sixth form college called Tom Wolf; he is 46 years old, he teaches A level ICT, Diploma ICT and GCSE ICT. Tom has few degrees in cumputer/ICT related subjects, so he is very experienced with computers, although not with programming. Currently Tom does not have good software to store and calculate data about his students. He needs to store data for approximately 250 students, so Tom requires an electronic equivalent of a markbook. Tom teaches GCSE, diploma and A level ICT. Therefore he requires the markbook to take in A-E grades for A level ICT and pass/merit/destinction/fail for GCSE and diploma ICT. The markbook needs to take in attendance and marks for units. The program must be flexible to create an infinite amount of units and must have an option of deleting units. Units should contain assignments, number of assignments within the unit should also have a possibilty of adding additional assignments without limitation. The program should have the database seperately from the main program, 
so that the databese could be backed up incase of hardware failures on seperate computers, hardrives or online backing-up website, such as google drive, which Tom is currently familiar with.
Also, another note is that Tom does not require a markbook to either mark students as on-time/late/absent, he already owns software that is capable enough of doing this.

\subsection{Define the current system} 
Currently Tom either uses google drive to store data or uses manual paper based system to store student's attendance, grades, predicted grades. 

\subsection{Describe the problems}
Tom is required to calculate predicted grades for 250 students by himself and this requires his own time, so school does not pay him for this. So, for Tom this is not satisfactory as 250 students is a big amount and so it requires a lot of effort and time. Therefore, Tom requires software that will perform those calculations automatically. 
\subsection{Section appendix}

\section{Investigation}

\subsection{The current system}
\subsubsection{Data sources and destinations}
In the current system, after Tom marks his students' course work, assesments, etc. he then counts up the total mark and writes the grade in his markbook or google drive database. Then based on the mark scheme, he sets the grade (either in google drive or his markbook) and the predicted grade for the end of the year. Predicted grade has to be calculated every term, so this is a lot of effort as well (as there are 250 students in total).

\subsubsection{Algorithms}
At this point I am aware of the following algorithms:

Attendance


\subsubsection{Data flow diagram}

\subsubsection{Input Forms, Output Forms, Report Formats}

\subsection{The proposed system}

\subsubsection{Data sources and destinations}

\subsubsection{Data flow diagram}

\subsubsection{Data dictionary}

\subsubsection{Volumetrics}

\section{Objectives}

\subsection{General Objectives}

\subsection{Specific Objectives}

\subsection{Core Objectives}

\subsection{Other Objectives}

\section{ER Diagrams and Descriptions}

\subsection{ER Diagram}

\subsection{Entity Descriptions}

\section{Object Analysis}

\subsection{Object Listing}

\subsection{Relationship diagrams}

\subsection{Class definitions}

\section{Other Abstractions and Graphs}

\section{Constraints}

\subsection{Hardware}

\subsection{Software}

\subsection{Time}

\subsection{User Knowledge}

\subsection{Access restrictions}

\section{Limitations}

\subsection{Areas which will not be included in computerisation}
\subsection{Areas considered for future computerisation}

\section{Solutions}

\subsection{Alternative solutions}

\subsection{Justification of chosen solution}
