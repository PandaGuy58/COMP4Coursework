\chapter{Analysis}

\section{Introduction}

\subsection{Client Identification} 
My client is Tom Wolf, he is 46 and he is my A level ICT teacher. Tom teaches A level ICT, Doploma ICT and GCSE ICT at Long Road sixth form college. As he is an ICT teacher, he is very educated in computer related subjects, but his programming experience is relatively small and he is not able to write complex programs programs for himself

\subsection{Define the current system} 
Currently the system is Tom himself and Google Drive and/or his hard drive on his laptop is used to store calculated data. Tom is required to collect grades of student from the school system, use an algorithm to calculate his/her predicted grade, create a table using already purcheased software such as Microsoft Word and  insert new calculated data to the table. This is very time consuming, as he is required to repeat this process after few assignments.

\subsection{Section appendix}

Interview Questions

What sort of system do you need currently?
- It has to be an electronic equivalent of a markbook
-It should be able to store infinite amount of units 
-Within the units should be assignments
-Assignments represent exams, work, tests grades
-The number of assignments within a unit also shouldn't have limitation
-Predicted grade should be calculated from all assignments from the markbook for a specific student


\subsection{Describe the problems}

Tom is required to calculate predicted grades for 200 students by himself and this requires his own time, so the school does not pay him for this. So, for Tom this is not satisfactory as 200 students is a big amount and so it requires a lot of effort and time. Therefore, Tom requires software that will perform those calculations automatically. 
\subsection{Section appendix}

\section{Investigation}

\subsection{The current system}

\subsubsection{Data sources and destinations}

In the current system, after Tom marks his students' course work, assesments, etc., he then has to count up the total marks and write the grade in his markbook or google drive database. Then based on the mark scheme, he sets the grade (either in google drive or his markbook) and the predicted grade for the end of the year. Predicted grade has to be calculated every term, so this is a lot of effort as well (as there are 200 students in total).

\subsubsection{Algorithms}
At this point I am aware of the following algorithms:

Attendance


\subsubsection{Data flow diagram}

\subsubsection{Input Forms, Output Forms, Report Formats}

\subsection{The proposed system}

\subsubsection{Data sources and destinations} 

\subsubsection{Data flow diagram}

\subsubsection{Data dictionary}

\subsubsection{Volumetrics}

\section{Objectives}

\subsection{General Objectives}

\subsection{Specific Objectives}

\subsection{Core Objectives}

\subsection{Other Objectives}

\section{ER Diagrams and Descriptions}

\subsection{ER Diagram}

\subsection{Entity Descriptions}

\section{Object Analysis}

\subsection{Object Listing}

\subsection{Relationship diagrams}

\subsection{Class definitions}

\section{Other Abstractions and Graphs}

\section{Constraints}

\subsection{Hardware}

\subsection{Software}

\subsection{Time}

\subsection{User Knowledge}

\subsection{Access restrictions}

\section{Limitations}

\subsection{Areas which will not be included in computerisation}
\subsection{Areas considered for future computerisation}

\section{Solutions}

\subsection{Alternative solutions}

\subsection{Justification of chosen solution}

Tom is required to calculate predicted grades for 250 students by himself and this requires his own time, so school does not pay him for this. So, for Tom this is not satisfactory as 250 students is a big amount and so it requires a lot of effort and time. Therefore, Tom requires software that will perform those calculations automatically. 


