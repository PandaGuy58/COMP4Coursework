 \chapter{Design}

\section{Overall System Design}

\subsection{Short description of the main parts of the system}

\subsection{System flowcharts showing an overview of the complete system}

\section{User Interface Designs}

\section{Program Structure}

\subsection{Top-down design structure charts}

\subsection{Algorithms in pseudo-code for each data transformation process}

\subsection{Object Diagrams}

\subsection{Class Definitions}

\section{Prototyping}

\section{Definition of Data Requirements}

\subsection{Identification of all data input items}
\begin{itemize}
    \item Student Name
    \item Student Surname
    \item Class Name
    \item Maximum Possible Assignment Mark
    \item Achieved Assignment Mark
    \item Assignment Name
    \item Unit Name

\subsection{Identification of all data output items}
    \item Assignment Percentage Mark
    \item Assignment Grade
    \item End of the Year Predicted Grade
    \item End of Unit Predicted Grade


\subsection{Explanation of how data output items are generated}
\begin{center}
     \begin{tabular}{|p{4}|p{4}|p{8}}
          \hline
          \textbf{Output Item}  & \textbf{Required Items} & \textbf{Description} \\ \hline
Assignment Achieved Grade & Percentage Grade & Achieved Grade is calculated from the Percentage grade, which is calculated first. Achieved Grade is calculated using the Achieved Grade Algorithim.
\hline
Percentage Grade & Assignment Maximum Mark, Assignment Achieved Mark & Percentage Grade = (Assignment Achieved Mark / Assignment Maximum Mark) * 100
\hline
End of the Year Predicted Grade & All End of Unit Assignments & This will be calculated using End of the Year Predicted Grade Algorithm.
\hline 
End of Unit Predicted Grade & All Assignment Grades & This required all Assignment Grades from the same unit to calculate the Predicted Grade for that unit
\line
\subsection{Data Dictionary}

\subsection{Identification of appropriate storage media}

For the storage media I am going to use a hard drive because it is very cheap, (not finished

\section{Database Design}

\subsection{Normalisation}

\subsubsection{ER Diagrams}

\subsubsection{Entity Descriptions}

\subsubsection{1NF to 3NF}

\section{Security and Integrity of the System and Data}

\subsection{Security and Integrity of Data}

\subsection{System Security}

\section{Validation}

\section{Testing}

\begin{landscape}
\subsection{Outline Plan}

\begin{center}
    \begin{tabular}{|p{2cm}|p{5cm}|p{5cm}|p{4cm}|}
        \hline
        \textbf{Test Series} & \textbf{Purpose of Test Series} & \textbf{Testing Strategy} & \textbf{Strategy Rationale}\\ \hline
        Example & Example & Example & Example \\ \hline
    \end{tabular}
\end{center}

\subsection{Detailed Plan}

\begin{center}
    \begin{longtable}{|p{1.5cm}|p{2.5cm}|p{2.5cm}|p{2cm}|p{2cm}|p{2cm}|p{2cm}|p{2cm}|}
        \hline
        \textbf{Test Series} & \textbf{Purpose of Test} & \textbf{Test Description} & \textbf{Test Data} & \textbf{Test Data Type (Normal/ Erroneous/ Boundary)} & \textbf{Expected Result} & \textbf{Actual Result} & \textbf{Evidence}\\ \hline
        Example & Example & Example & Example & Example & Example & Example & Example \\ \hline
    \end{longtable}
\end{center}
\end{landscape}
